
\documentclass{article}

\usepackage[francais]{babel}
\usepackage[T1]{fontenc}
\usepackage[utf8]{inputenc}
\usepackage{a4wide}
\usepackage{palatino}

\let\bfseriesbis=\bfseries \def\bfseries{\sffamily\bfseriesbis}


\newenvironment{point}[1]%
{\subsection*{#1}}%
{}

\setlength{\parskip}{0.3\baselineskip}

\begin{document}
\pagenumbering{gobble}

\title{Generalization of counterexamples for inductive invariant synthesis}

\author{Mickaël Laurent, supervised by Bryan Parno\\Carnegie Mellon University}

\date{March-July 2018}

\maketitle

\pagestyle{empty} %
\thispagestyle{empty}

%% Attention: pas plus d'un recto-verso!


\begin{point}{General context}
  
  As distributed programs become more and more widespread, their verification is a major challenge.
  The general purpose of my internship is to make the certification of those programs easier, by providing new methods and semi-automated tools that will assist the user in this process.
  In this report, we are using IVy, a language combined with a set of tools that can be used to write certified distributed protocols.
  The particularity of IVy is that it restricts the language and the logic used for the specification in order to ensure some decidability properties.
  In particular, checking whether an inductive invariant is correct or not become decidable.

  When writing a protocol using IVy, the two main challenges are
  \begin{itemize}
    \item To implement the protocol and specify it by staying in the decidable fragment imposed by IVy,
    \item Once the previous point is done, to find an inductive invariant.
  \end{itemize}

  During this internship, I have been working on the second point.

\end{point}

\begin{point}{The problem}
  
  %The problem we want to solve is the following:
  %given an invariant that is not inductive, we want to strenghten it by generating a new (correct) invariant as strong as possible, if possible without breaking the decidability of the system.

  Quelle est la question que vous avez résolue? Pourquoi est-elle
  importante, à quoi cela sert-il d'y répondre?  Pourquoi êtes-vous
  le premier chercheur de l'univers à l'avoir posée?

\end{point}

\begin{point}{Contributions}

  Qu'avez vous proposé comme solution à cette question? Attention, pas
  de technique, seulement les grandes idées! Soignez particulièrement
  la description de la démarche \emph{scientifique}.
 
\end{point}

\begin{point}{Results and guarantees}

  Qu'est-ce qui montre que cette solution est une bonne solution? Des
  expériences, des corollaires? Commentez la \emph{stabilité} de votre
  proposition: comment la validité de la solution dépend-elle des
  hypothèses de travail?

\end{point}


\begin{point}{Review and prospects}
  
  Et après? En quoi votre approche est-elle générale? Qu'est-ce que
  votre contribution a apporté au domaine? Que faudrait-il faire
  maintenant? Quelle est la bonne \emph{prochaine} question?

\end{point}


\end{document}







